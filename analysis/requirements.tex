\section{Requirement Specification}
We have read the client description for the User-Administration Module, in the following called the program, and based on that made the following requirement specification.

\subsection{Functional requirements}
\begin{adjustwidth}{2em}{0pt}
    
    \subsubsection*{FU01: Add user}
    \begin{adjustwidth}{1em}{0pt}
        \textbf{Description:}
        The operator shall be able to add new users to the program. The program shall receive the following information about the new user:
        \begin{multicols}{2}
            \begin{itemize}
                \item User ID (PrimaryKey)
                \item Username (2-20 characters)
                \item Initials (2-4 characters)
                \item CPR (10 characters)
                \item Roles (one of the following: Admin, Pharmacist, Foreman, Operator)
            \end{itemize}
        \end{multicols}
        \noindent\textbf{Use Case Flow:}
        Preconditions: None.\\
        1. The operator chooses "Add user" in the menu.\\
        2. The operator is prompted for user ID and enters this.\\
        3. The operator is prompted for a username and enters this.\\
        4. The operator is prompted for initials and enters this.\\
        5. The operator is prompted for CPR and enters this.\\
        6. The operator is prompted for a role and chooses the desired one.\\
        Postconditions: The new user is added to the program.\\
        \textbf{Requirement references:}
        FU05.
    \end{adjustwidth}
    
    \subsubsection*{FU02: Show users}
    \begin{adjustwidth}{1em}{0pt}
        \textbf{Description:}
        The operator shall be able to show all users registered in the program.\\
        \textbf{Use Case Flow:}
        Preconditions: None.\\
        1. The operator chooses "Show users" in the menu.\\
        Postconditions: The operator is shown a comma-separated list of user IDs and their corresponding name.\\
        \textbf{Requirement references:}
        None.
    \end{adjustwidth}
    
    \subsubsection*{FU03: Update user}
    \begin{adjustwidth}{1em}{0pt}
        \textbf{Description:}
        The operator shall be able to update information on the users.\\
        \textbf{Use Case Flow:}
        Preconditions: None.\\
        1. The operator chooses "Update user" in the menu.\\
        2. The operator is prompted to enter the user ID of the user he wants to update.\\
        3. The operator is prompted to enter the field he wants to update.\\
        4. The operator is prompted for the new value of the chosen field and enters this.\\
        Postconditions: The chosen field of the chosen user is updated.\\
        \textbf{Requirement references:}
        RE01.
    \end{adjustwidth}
    
    \subsubsection*{FU04: Remove user}
    \begin{adjustwidth}{1em}{0pt}
        \textbf{Description:}
        The operator shall be able to remove a user.\\
        \textbf{Use Case Flow:}
        Preconditions: None.\\
        1. The operator is prompted to enter the user ID of the user he wants to delete.\\
        2. The operator is prompted to verify his action, by clicking "Yes, I am sure I want to delete this user".\\
        Postconditions: The user is removed from the program.\\
        \textbf{Requirement references:}
        None.
    \end{adjustwidth}
    
    \subsubsection*{FU05: Password generation}
    \begin{adjustwidth}{1em}{0pt}
        \textbf{Description:}
        The program shall generate a password for the user automatically upon creation.\\
        \noindent\textbf{Use Case Flow:}
        Preconditions: \\
        1. \\
        Postconditions: \\
        \textbf{Requirement references:}
        RE01.
    \end{adjustwidth}
    
\end{adjustwidth}

\subsection{Usability requirements}
\begin{adjustwidth}{2em}{0pt}

    \subsubsection*{US01: Main Menu}
    \begin{adjustwidth}{1em}{0pt}
        \textbf{Description:}
        The program shall consist of a menu with 5 items: Add user, Show users, Update user, Remove user and Close program.\\
        %\textbf{Use Case Flow:}
        %1. The operator is prompted with a selection of inputs, to be written into the TUI.\\
        %2. The operator will choose one of these selections from the menu.\\
        %3. The operator selects create user and prompts the TUI.\\
        \textbf{Requirement references:}
        FU01, FU02, FU03, FU04.
    \end{adjustwidth}
    
\end{adjustwidth}

\subsection{Reliability requirements}
\begin{adjustwidth}{2em}{0pt}

    \subsubsection*{RE01: Password}
    \begin{adjustwidth}{1em}{0pt}
        \textbf{Description:}
        The password shall contain at least 6 characters of at least 3 of the following 4 categories: 
        \begin{itemize}
            \item lower-case letters ('a' to 'z')
            \item capital letters ('A' to 'Z')
            \item numbers ('0' to '9')
            \item special  characters ('.', '-', '\_', '+', '!', '?', '=')
        \end{itemize}
        \textbf{Requirement references:}
        FU05.
    \end{adjustwidth}

    \subsubsection*{RE02: CPR Validation}
    \begin{adjustwidth}{1em}{0pt}
        \textbf{Description:}
        There shall be validation on the first six digits of the CPR number, to check if it is a valid date. The length of the CPR number shall also be validated to be 10 digits.\\
        \textbf{Requirement references:}
        None.
    \end{adjustwidth}

    \subsubsection*{RE03: UserID Validation}
    \begin{adjustwidth}{1em}{0pt}
        \textbf{Description:}
        The user id shall be an integer between 11 and 99 and has to be unique for the user.\\
        \textbf{Requirement references:}
        None.
    \end{adjustwidth}
    
\end{adjustwidth}

\subsection{Performance requirements}
\begin{adjustwidth}{2em}{0pt}

    None.

\end{adjustwidth}

\subsection{Supportability requirements}
\begin{adjustwidth}{2em}{0pt}

    \subsubsection*{SU01: Translation}
    \begin{adjustwidth}{1em}{0pt}
        \textbf{Description:}
        The program shall be translatable.\\
        \textbf{Requirement references:}
        None.
    \end{adjustwidth}

\end{adjustwidth}
