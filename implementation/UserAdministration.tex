\section{Administrating Users}

Using the \texttt{UserAdministrationDB} class it is easy to manage users in the database. There are a few essential methods in this class that makes it a lot easier to manage all the users in the database.

First step to manage your users is to list all users in the database. The \texttt{getUsersList} method does just that. It returns a list of users. Using it is quite simple:

\begin{listing}[ht]
\begin{minted}[frame=lines,linenos]{java}
public static void main(String[] args) {
    DBConnection dbConnection = new DBConnection(
        "localhost", 3306, "my-db", "root", "root-Passw0rd!"
    );
    
    UserAdministrationDB userAdm = new UserAdministrationDB(dbConnection);
    
    List<User> users;
    try {
        users = userAdm.getUserList();
    } catch(IUserAdministration.DataAccessException e) {
        System.err.prinln(e.getMessage());
        e.printStackTrace();
        return;
    }
    
    users.stream().forEach(System.out::println);
}
\end{minted}
\caption{Get a List of Users}
\label{listing:3}
\end{listing}

It is easy to see that it is a lot easier to use the database using the \texttt{UserAdministrationDB} class.

One of the cool features of this class is when it comes to editing existing users. Lets say we have the following user:

\begin{multicols}{2}
    \begin{itemize}
        \item \textbf{ID}: 11
        \item \textbf{Username}: My User
        \item \textbf{Initials}: myus
        \item \textbf{Roles}: Admin, Foreman
        \item \textbf{CPR}: 1010901235
        \item \textbf{Password}: l-m2Ehdc
        \
    \end{itemize}
\end{multicols}

Say we want to edit the username to \textbf{John Doe}, initials to \textbf{jodo}. Remove the \textbf{Admin} role and instead give him the \textbf{Operator} role, and then lastly, generate a new password for him, we could do it in a quite neat way. The way the \texttt{updateUser} method in \texttt{UserAdministrationDB} works is, it takes a user in as an argument, then pulls the "\textit{old}" user by his \textit{id} from the database, compares all values, and then only changing what has been changed. Here is an example:

\begin{listing}[ht]
\begin{minted}[frame=lines,linenos]{java}
public static void main(String[] args) {
    DBConnection dbConnection = new DBConnection(
        "localhost", 3306, "my-db", "root", "root-Passw0rd!"
    );
    
    UserAdministrationDB userAdm = new UserAdministrationDB(dbConnection);
    List<String> roles = new ArrayList<>();
    roles.add("Admin"); roles.add("Foreman");
    
    User myUser = new User(11, "My User", "myus", roles, "1010901235");
    
    // Now, let's change the user
    user.setUserName("John Doe");
    user.setInitials("jodo");
    user.removeRole("Foreman");
    user.addRole("Operator");
    user.generateNewPassword();
    
    // Now just update the user in the database
    try {
        userAdm.updateUser(myUser); 
    } catch (IUserAdministration.DataAccessException e) {
        System.err.println(e.getMessage());
    }
}
\end{minted}
\caption{Editing a User}
\label{listing:4}
\end{listing}

\texttt{UserAdministrationDB} has a quite is detecting changes by comparing the user given in the argument with the current user (with the same id) in the database:

\begin{listing}[ht]
\begin{minted}[frame=lines,linenos]{java}
public void updateUser(User user) throws DataAccessException {
    final User oldUser = getUser(user.getUserId());
    
    // This is to keep an eye on the count of changes
    int changeCount = 0;
    // We are not defining the whole statement yet, 
    // because we still do not know what changes has been made.
    String sql = "UPDATE user SET";
    
    if (!oldUser.getUserName().equals(user.getUserName())) {
        // If the changeCount is greater than zero, we need to add a comma
        // to seperate each field that gets updated
        if (changeCount > 0) sql += ", ";
        changeCount++;
        
        // Note that we are not using any prepared statements
        sql += " username = '" + user.getUserName() + "'";
    }
    if (...) {
        ...
    }    
}
\end{minted}
\caption{Editing a User}
\label{listing:4}
\end{listing}

This way, it is way easier to manage users and use this class.