\section{Manual testing}
\subsection{Negative and positive test}
Through this project the testing have been done manually. This implies opening the program, and test all the features that is implemented. In this specific project there are five overall features, create user, show users, edit user, delete user and exit program. Every one of the features were tested, in two different ways. The first testing method is called a positive test, this is a method where you are giving your program the input it is expecting. The goal is to see if every works as expected. So in the program we did go through every general feature, and gave it the input it wanted us to give it.\\
The second testing method that was also used in this project, is called a negative test. What is a negative test? A negative test is the exact opposite of a positive test. Instead of giving the program the input that it's expecting you'll try to break the program. Does the program want a number, then you give it some text. That is the basics of a negative test. Then doing negative tests on this program, the program got some unexpected arguments, and handled it fine. 
\subsection{Boundary test}
Through out the project we were provided some criteria, that must to be observed. Some of those involved boundaries of the users information. This was for example the username, initial, userid, etc. maximum and minimum length. So it was obvious to use boundary testing, in this case. Boundary testing is then you are testing your program to the edges, pushing a variable to it's maximum, minimum and middle values. It would also be a optimal to try values way over and way lower than what is accepted by the program. By doing this, you get a closer look at how your program is handling such types of errors.

\section{JUnit testing}
\subsection{Database test}
We have made a test of the database insert, update and delete statements in the \texttt{DBTester} class. This test tries to create a user in the database to see if that works, and then afterwards try to add the same user again, to see if the program will throw an error and cancel the statement. Then a second user is created, and his username is updated. Finally the first user is deleted. The test has no assertions, but will fail if an uncatched exception is thrown.\\
The tests succeeded, which means that the database connection and the SQL-queries should be working as intended.